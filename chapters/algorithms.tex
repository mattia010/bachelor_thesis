\chapter{Algoritmi su WFF come stringhe binarie}

\section{Stabilire se una clausola è soddisfatta da un assegnamento}
Una clausola è soddisfatta da un assegnamento se e solo se almeno uno 
dei suoi letterali risulta vero.
\begin{algorithm}[H]
    \caption{Algoritmo per stabilire se una clausola è soddisfatta da un assegnamento.}
    \label{alg:is_clause_sat}
    \begin{algorithmic}[1]
        \Procedure{Is-clause-satisfied}{clause, assignment}
            \If{assignment.superposVarNum $\neq$ 0}
                \State \textbf{error}
            \EndIf
            \If{clause.length $>$ assignment.length}
                \State \textbf{error}
            \EndIf
            \For{i $\gets$ to clause.length - 1}
                \If{clause.vars[i] $\ge$ assignment.length}
                    \State \textbf{error}
                \EndIf
            \EndFor\\

            \For{i $\leftarrow$ 0 to clause.length - 1} 
                \State varIndex $\gets$ clause.vars[i]
                \State varValue $\gets$ assignment.varValues[varIndex]
                \State litValue $\gets$ varValue $\oplus$ clause.isNeg[i]
                \If{litValue}
                    \State \Return true 
                \EndIf
            \EndFor
            \State \Return false
        \EndProcedure
    \end{algorithmic}
\end{algorithm}
Questa procedura ha complessità costante $\Theta(1)$, in quanto $k$ (e quindi clause.length) è costante: qualunque sia il numero 
delle variabili su cui è definito l'assegnamento, ogni chiamata, nel caso peggiore, esegue il 
corpo del ciclo $k$ volte.

\section{Trovare le clausole soddisfatte da un assegnamento}
Il problema di trovare le clausole verificate da un assegnamento può anche essere definito
come trovare quali delle possibili $2^k \binom{n}{k}$ $k$-clausole definibili sulle $n$
variabili dell'assegnamento risultano vere.\\
Questa procedura riceve un assegnamento di lunghezza $n$ 
e un intero $k$ che identifica a quale problema $k$-SAT si
sta facendo riferimento, e restituisce una WFF di lunghezza $2^k \cdot \binom{n}{k}$,
in cui l'$i$-esimo bit viene impostato a 1 se la clausola di indice
$i$ risulta soddisfatta, 0 altrimenti.
\begin{algorithm}[H]
    \caption{Algoritmo per trovare tutte le clausole soddisfatte da un assegnamento.}
    \label{alg:find_sat_clauses}
    \begin{algorithmic}[1]
        \Procedure{Find-sat-clauses}{assignment, k}  
            \If{assignment.superposVarNum $\neq$ 0}
                \State \textbf{error}
            \EndIf
            \If{k $>$ assignment.length}
                \State \textbf{error}
            \EndIf\\

            \State varNum $\gets$ assignment.length
            \State satClauses $\gets$ \Call{Construct-wff}{varNum, k}
            \State tmpClause $\gets$ \Call{Construct-clause}{k}\\

            \For{i $\gets$ 0 to satClauses.length - 1}
                \If{\Call{Is-clause-satisfied}{tmpClause, assignment}}
                    \State \Call{Add-clause-to-wff}{satClauses, tmpClause, i}
                \EndIf
                \State \Call{Increment-clause}{tmpClause, varNum}
            \EndFor
            \State \Return satClauses
        \EndProcedure
    \end{algorithmic}
\end{algorithm}
Le chiamate a \textsc{Is-clause-satisfied}, \textsc{Increment-clause} e \textsc{Add-clause-to-wff} hanno complessità costante.
Il corpo del ciclo viene invece eseguito $2^k \binom{n}{k}$ volte, 
dove $n$ è il numero di variabili. Ogni esecuzione del corpo
corrisponde alla valutazione di una delle possibili clausole definibili
su $n$ variabili.\\
Di conseguenza, la complessità dell'algoritmo è $\Theta (n^k)$.


\section{Trovare le WFF soddisfatte da un assegnamento}
Generare ogni WFF per poi verificare la sua soddisfacibilità sarebbe 
poco efficiente, in quanto il tempo di calcolo sarebbe \textit{esponenziale nel numero
di possibili clausole}.\\
Viene quindi presentato un algoritmo per trovare le WFF soddisfatte
\textit{esponenziale nel numero di clausole soddisfatte dall'assegnamento}.\\
\\
Il numero di WFF soddisfatte da un assegnamento è pari a $2^{\bra{2^k - 1} \binom{n}{k}}$
e ogni WFF richiede, considerando solo la stringa binaria che la rappresenta,
un numero di bit pari a $2^k \cdot \binom{n}{k}$.
Se si memorizzassero quindi tutte le WFF soddisfatte, per poi stamparle, servirebbe
una quantità di memoria molto grande: per un problema $5$-SAT su 30 variabili, ad
esempio, servirebbero $2^{4417643} \cdot 4417686$ TB.
Per risolvere questo problema, quindi, una WFF viene immediatamente 
stampata e la stessa area di memoria viene modificata per 
ottenere la prossima WFF soddisfatta, che a sua volta verrà stampata e così via.
\begin{algorithm} [H]
    \caption{Algoritmo per trovare le WFF soddisfatte da un assegnamento}
    \label{alg:find_sat_wff}
    \begin{algorithmic}[1]
        \Procedure{Find-satisfied-wff}{assignment, k}
            \If{assignment.superposVarNum $\neq$ 0}
                \State \textbf{error}
            \EndIf
            \If{k $>$ assignment.length}
                \State \textbf{error}
            \EndIf\\

            \State varNum $\gets$ assignment.length
            \State satClauses $\gets$ \Call{Find-sat-clauses}{assignment, k}
            \State clausesNum $\gets$ satClauses.length
            \State satClausesNum $\gets$ satClauses.presentClausesNum
            \State tmpWff $\gets$ \Call{Construct-wff}{varNum, k}
            \State tmpCl $\gets$ \Call{Construct-clause}{k}\\
            
            \For{j $\gets$ 0 to \Call{Pow}{2, satClausesNum} - 1}
                \State \Call{Print-wff}{wff}
                \State \Call{Reset-clause}{tmpCl}
                \For{i $\gets$ 0 to clausesNum - 1}
                    \If{satClauses.isClausePresent[i]}
                        \If{tmpWff.isClausePresent[i]}
                            \State \Call{Remove-clause-from-wff}{tmpWff, tmpCl, i} 
                        \Else                    
                            \State \Call{Add-clause-to-wff}{tmpWff, tmpCl, i}
                            \State \textbf{break}
                        \EndIf
                    \EndIf
                    \State \Call{Increment-clause}{tmpCl, varNum}
                \EndFor
            \EndFor
        \EndProcedure
    \end{algorithmic}
\end{algorithm}
Le chiamate a \textsc{Reset-clause}, \textsc{Remove-clause-from-wff}, \textsc{Add-clause-to-wff} e \textsc{Increment-clause} hanno complessità costante.\\
Detto $c$ il costo della chiamata
a \textsc{Print-wff}, essa deve essere eseguita su esattamente $2^{\bra{2^k - 1}\binom{n}{k}}$ WFF. Detto invece $r_i$ il numero di bit impostati a 1 a destra dell'$i$-esimo bit, su esso verrà effettuato un numero di controlli pari a:
\begin{equation*}
    \frac{2^{\bra{2^k - 1} \binom{n}{k}}}{2^{r_i}}
\end{equation*}
Il caso peggiore si verifica quando tutti i $\binom{n}{k}$ bit impostati a 0 sono quelli meno significativi della WFF, e la sua complessità è pari a:
\begin{equation*}
    O \bra{2^{\bra{2^k - 1}\binom{n}{k}} \cdot \bra{c + \binom{n}{k}} 
    + \sum_{i=1}^{(2^k - 1)\binom{n}{k}} \frac{2^{\bra{2^k - 1}\binom{n}{k}}}{2^{i-1}}} 
    = O\bra{2^{O\bra{n^k}} \cdot \bra{c + O\bra{n^k}}}
\end{equation*}
Questo algoritmo può essere in realtà utilizzato per effettuare un'operazione
qualsiasi su tutte le WFF soddisfatte dall'assegnamento: basta infatti sostituire
la riga 15 con l'operazione desiderata.

\section{Stabilire quali clausole di una WFF sono soddisfatte da un assegnamento}
Per stabilire quali clausole di una WFF di un problema $k$-SAT sono soddisfatte da un assegnamento su $n$ variabili
bisogna utilizzare un algoritmo molto simile a quello per trovare le clausole generiche
soddisfatte da un assegnamento, al quale però viene aggiunta la condizione che una 
clausola verificata debba anche essere presente nella WFF fornita.
Anche in questo caso viene restituita una stringa binaria $\langle x_0, \ldots, x_{m - 1} \rangle$, con
$m = 2^k \cdot \binom{n}{k}$, il cui $i$-esimo bit è:
\begin{equation*}
    x_i = \begin{cases}
        1 & \tn{clausola di indice } i \tn{ compare nella WFF } \land \tn{ clausola è soddisfatta} \\
        0 & \tn{altrimenti}
    \end{cases}
\end{equation*}
\begin{algorithm}[H]
    \caption{Algoritmo per trovare tutte le clausole di una WFF soddisfatte da un assegnamento.}
    \label{alg:find_wff_sat_clauses}
    \begin{algorithmic}[1]
        \Procedure{Find-wff-sat-clauses}{wff, assg}
            \If{wff.varNum $\neq$ assg.length}
                \State \textbf{error}
            \EndIf
            \If{assg.superposVarNum $\neq$ 0}
                \State \textbf{error}
            \EndIf
            \If{wff.kSat $>$ assg.length}
                \State \textbf{error}
            \EndIf\\

            \State k $\gets$ wff.kSat 
            \State varNum $\gets$ assg.length
            \State wffSatClauses $\gets$ \Call{Construct-wff}{varNum, k}
            \State tmpCl $\gets$ \Call{Construct-clause}{k}\\

            \For{i $\gets$ 0 to wffSatClauses.length - 1}
                \If{wff.isClausePresent[i] $\land$ \Call{Is-clause-satisfied}{tmpCl, assg}}
                    \State \Call{Add-clause-to-wff}{wffSatClauses, tmpCl, i}
                \EndIf
                \State \Call{Increment-clause}{tmpCl, varNum}
            \EndFor
            \State \Return wffSatClauses
        \EndProcedure
    \end{algorithmic}
\end{algorithm}
Le chiamate a \textsc{Add-clause-to-wff} e \textsc{Increment-clause} hanno complessità costante. Di conseguenza, la complessità dell'algoritmo è polinomiale e pari a:
\begin{equation*}
    \Theta \bra{2^k \binom{n}{k}} = \Theta(n^k)
\end{equation*}

\section{Stabilire se una WFF è soddisfatta da un assegnamento}
Una WFF è soddisfatta da un assegnamento se e solo se tutte le sue clausole sono soddisfatte.
Si utilizza un algoritmo simile a quello per trovare quali clausole
di una WFF sono soddisfatte da un assegnamento, ma se una clausola presente
nella WFF non risulta soddisfatta si restituisce immediatamente false.
\begin{algorithm}[H]
    \caption{Algoritmo per stabilire se una WFF è soddisfatta da un dato assegnamento}
    \label{alg:is_wff_satisfied}
    \begin{algorithmic}
        \Procedure{Is-wff-satisfied}{wff, assg}
            \If{wff.varNum $\neq$ assg.length}
                \State \textbf{error}
            \EndIf
            \If{assg.superposVarNum $\neq$ 0}
                \State \textbf{error}
            \EndIf
            \If{wff.kSat $>$ assg.length}
                \State \textbf{error}
            \EndIf\\

            \State k $\gets$ wff.kSat
            \State varNum $\gets$ assg.length
            \State tmpCl $\gets$ \Call{Construct-clause}{k}\\

            \For{i $\gets$ 0 to wff.length - 1}
                \If{wff.isClausePresent[i] $\land$ !\Call{Is-clause-satisfied}{tmpCl, assg}}
                    \State \Return false
                \EndIf
                \State \Call{Increment-clause}{tmpCl, varNum}
            \EndFor
            \State \Return true
        \EndProcedure
    \end{algorithmic}
\end{algorithm}
Le chiamate a \textsc{Is-clause-satisfied} e \textsc{Increment-clause} hanno complessità
costante e, di conseguenza, l'algoritmo ha complessità polinomiale:
\begin{equation*}
    O \bra{2^k \binom{n}{k}} = O \bra{n^k}
\end{equation*}


\section{Ridurre la dimensione di una WFF}
Utilizzando una rappresentazione binaria per le WFF, è possibile che una variabile sia
considerata appartenente ad essa ma che in realtà nessuna clausola che la contiene compaia 
effettivamente nella WFF: essa può quindi essere rimossa, diminuendo il numero 
totale di variabili su cui la WFF è definita e di conseguenza la sua dimensione.
La rimozione comporta, in pratica, l'accorciamento della WFF, effettuato rimuovendo
tutti i bit che rappresentano clausole contenenti almeno un letterale su una variabile assente nella 
WFF.
Siccome le variabili considerate non compaiono nella WFF, tutti i bit che verranno rimossi
saranno uguali a 0.\\
Se $r$ è il numero di variabili che possono essere rimosse, la WFF accorciata
avrà un numero di possibili clausole pari a:
\begin{equation*}
    m = 2^k \cdot \binom{n-r}{k}
\end{equation*}

\begin{algorithm}[H]
    \caption{Algoritmo per contare il numero di variabili rimuovibili.}
    \label{alg:count_rem_vars}
    \begin{algorithmic}[1]
        \Procedure{Count-removable-vars}{wff}
            \State r $\gets$ 0
            \For{i $\gets$ 0 to wff.varNum - 1}
                \If{wff.litCount[i] + wff.litCount[i + wff.varNum] = 0}
                    \State ++r
                \EndIf
            \EndFor
            \State \Return r
        \EndProcedure
    \end{algorithmic}
\end{algorithm}

\begin{algorithm}[H]
    \caption{Algoritmo per verificare se una clausola è rimuovibile.}
    \begin{algorithmic}[1]
        \Procedure{Is-clause-removable}{clause, wff}
            \If{clause.length $\neq$ wff.kSat}
                \State \textbf{error}
            \EndIf
            \For{i $\gets$ 0 to clause.length - 1}
                \If{clause.vars[i] $\ge$ wff.varNum}
                    \State \textbf{error}
                \EndIf
            \EndFor\\

            \For{i $\gets$ 0 to clause.length - 1}
                \State var $\gets$ clause.vars[i]
                \If{wff.litCount[var] + wff.litCount[var + wff.varNum] = 0}
                    \State \Return true
                \EndIf
            \EndFor
            \State \Return false
        \EndProcedure
    \end{algorithmic}
\end{algorithm}

\begin{algorithm}[H]
    \caption{Algoritmo per accorciare una WFF}
    \label{alg:shorten_wff}
    \begin{algorithmic}[1]
        \Procedure{Shorten-wff}{wff}
            \State r $\gets$ \Call{Count-removable-vars}{wff}
            \If{r = 0}
                \State \Return
            \EndIf\\

            \State k $\gets$ wff.kSat
            \State newVarNum $\gets$ wff.varNum - r
            \State shortWff $\gets$ \Call{Construct-wff}{newVarNum, k}
            \State clause $\gets$ \Call{Construct-clause}{k}\\

            \State tmp $\gets$ 0
            \State newLitCount $\gets$ nuovo array di interi di lunghezza 2 $\cdot$ newVarNum
            \For{i $\gets$ 0 to wff.varNum - 1}
                \If{wff.litCount[i] + wff.litCount[i + wff.varNum] $\neq$ 0}
                    \State newLitCount[tmp] $\gets$ wff.litCount[i]
                    \State newLitCount[tmp + newVarNum] $\gets$ wff.litCount[i +wff.varNum]
                    \State ++tmp
                \EndIf
            \EndFor
            \State shortWff.litCount $\gets$ newLitCount
            \State shortWff.presentClausesNum $\gets$ wff.presentClausesNum\\
            
            \State tmp $\gets$ 0
            \For{i $\gets$ 0 to wff.length - 1}
                \If{!\Call{Is-clause-removable}{clause, wff}}
                    \State shortWff.isClausePresent[tmp] $\gets$ wff.isClausePresent[i]
                    \State ++tmp
                \EndIf
                \State \Call{Increment-clause}{clause, wff.varNum}
            \EndFor
            \State wff $\gets$ shortWff
        \EndProcedure
    \end{algorithmic}
\end{algorithm}
Sia $n$ il numero di variabili su cui è definita la WFF. 
\textsc{Count-removable-vars} ha complessità lineare $\Theta(n)$, mentre
l'algoritmo \textsc{Is-clause-removable} ha complessità costante $\Theta(1)$.\\
Per quanto riguarda \textsc{Shorten-wff}, la sua complessità è pari a:
\begin{equation*}
    \Theta(n) + O(n) + O \bra{2^k \binom{n}{k}} = O \bra{n^k}
\end{equation*}
ed è quindi polinomiale nel numero di variabili.