\chapter{DIMACS CNF parser}
Durante l'implementazione dei precedenti algoritmi è
stato necessario trovare un modo di salvare e leggere WFF su file, in modo da poter 
memorizzare quelle considerate più interessanti o eseguire gli algoritmi su WFF generate
da altre persone.
Si è quindi scelto di sviluppare un piccolo parser basato sul formato DIMACS CNF \cite{dimacs-cnf}, 
standard de-facto nella rappresentazione delle CNF-WFF. La sua 
funzione è sia quella di convertire una WFF da formato binario a DIMACS CNF, per poi
salvarla su file, che leggere una WFF in DIMACS CNF da file e convertirla in formato 
binario, in modo tale da essere manipolabile dagli algoritmi di questo lavoro.\\
Dovendo gli algoritmi operare solo su WFF $k$-SAT in cui ogni clausola ha esattamente $k$ letterali, se il file presenta clausole di dimensione differente, il parser si occupa di renderle tutte della stessa dimensione, come descritto nella sezione 3.1.\\
Se una clausola presenta letterali opposti sulla stessa variabile, e non potendo rappresentare questo caso nella rappresentazione binaria adottata, allora la clausola, che è una tautologia, viene rimossa dalla WFF.
Se invece una clausola presenta più ripetizioni dello stesso letterale, solo un letterale viene mantenuto, mentre tutte le copie vengono scartate.\\
\\
Prima di descrivere il funzionamento del parser, è utile analizzare il formato DIMACS CNF.
Questo formato permette di rappresentare tutte le CNF-WFF, ovvero tutte quelle WFF che sono congiunzione di disgiunzioni di letterali.
Detto $n$ il numero di variabili presenti nella WFF, a ogni variabile viene associato un indice $i, 1 \le i \le n$. Un letterale positivo viene rappresentato come l'indice della variabile su cui è definito, mentre un letterale negativo viene rappresentato sempre dall'indice della variabile, il quale però viene preceduto dal simbolo "-" (meno).\\
Ogni clausola viene rappresentata invece elencando i suoi letterali, separati da uno spazio, e la sua terminazione viene indicata con il numero 0.
Solitamente, clausole diverse sono poste su righe differenti del file: il carattere separatore per le clausole rimane comunque 0, e spezzare una clausola su più righe non comporta alcun errore.\\
Molti parser richiedono inoltre un header, posto a inizio file, della forma:
\begin{center}
    \verb|p cnf <varNum> <clausesNum>|
\end{center}
che indica il numero di variabili su cui la WFF è definita e il numero di clausole presenti.\\
Opzionalmente, è possibile aggiungere dei commenti utilizzando la seguente sintassi:
\begin{center}
    \verb|c <comment>|
\end{center}
Esempio:\\
La codifica in DIMACS CNF di $(x_1 \lor$ ¬$x_4, \lor x_2) \land (x_1 \lor x_4 \lor x_5) \land ($¬$x_2 \lor $¬$x_3)$ è:\\
\verb|c questo è un commento|\\
\verb|p cnf 5 3|\\
\verb|1 -4 2 0|\\
\verb|1 4 5 0|\\
\verb|-2 -3 0|\\
\\
Per quanto riguarda il parser, il suo funzionamento può essere così riassunto:
\begin{enumerate}
    \item legge una riga da file;
    \item se ha letto end-of-file, allora passa al punto 8;
    \item altrimenti, salva in memoria la clausola letta sotto forma di vettore di letterali;
    \item ordina il vettore in modo tale da ottenere una clausola in forma "standard";
    \item verifica se è possibile rimuovere la clausola (letterali opposti e quindi tautologia) o se è possibile accorciarla (ripetizioni dello stesso letterale), ed eventualmente esegue la modifica su di essa;
    \item aggiunge poi il vettore che rappresenta la clausola a un vettore di clausole, che rappresenta la WFF;
    \item torna al punto 1;
    \item modifica le clausole presenti nel vettore delle clausole in modo che siano tutte di dimensione $k$;
    \item ordina il vettore delle clausole secondo la relazione d'ordine definita sulle clausole nella sezione 3.1;
    \item istanzia una WFF e, scorrendo il vettore delle clausole, imposta a 1 i bit che corrispondono alle clausole presenti nel vettore.
\end{enumerate}