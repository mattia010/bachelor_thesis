\chapter{Definizione del problema}

\section{Rappresentare gli oggetti}

\subsubsection{Rappresentare le WFF}
Il problema $k$-SAT è definito normalmente su WFF con clausole contenenti \textit{al massimo} $k$ letterali.
È possibile però, senza perdere di generalità, trattare solo le WFF con esattamente $k$ letterali
per clausola: una clausola con $u$ letterali, $u < k$, può essere infatti riscritta come una clausola logicamente equivalente di $k$ letterali aggiungendo $r = k - u$ letterali su $r$ nuove variabili non ancora presenti nella WFF, e obbligando questi letterali ad essere falsi.\\
Un letterale $l$ qualsiasi può essere facilmente reso falso aggiungendo $2^{k-1}$ clausole definite nel seguente modo:
\begin{enumerate}
    \item scelgo $k-1$ variabili già appartenenti alla WFF e i cui letterali non sono stati usati per aumentare la dimensione di un'altra clausola.
    \item aggiungo alla WFF la $k$-clausola $\bra{l_0, \ldots, l_{k-2}, \lnot l}$, contenente tutti i $k-1$ letterali positivi sulle variabili scelte e il letterale da rendere falso negato, e tutte le sue varianti sui primi $k-1$ letterali.
\end{enumerate}
Ad esempio, se $k=3$, si vuole rendere il letterale $l$ falso e sono state scelte due variabili $v_0$ e $v_1$ già presenti nella WFF, verranno aggiunte le clausole:
\begin{itemize}
    \item $\bra{v_0, v_1, \lnot l}$
    \item $\bra{\lnot v_0, v_1, \lnot l}$
    \item $\bra{v_0, \lnot v_1, \lnot l}$
    \item $\bra{\lnot v_0, \lnot v_1, \lnot l}$
\end{itemize}
Sia $C$ l'insieme delle $u$-clausole con $u < k$ e $u_i$ il numero di letterali presenti nella clausola $c_i \in C$, allora non è necessario definire un numero di nuove
variabili pari a $\sum_{i = 0}^{|C|-1} \bra{k-u_i}$ per renderle tutte di $k$ letterali:
le variabili utilizzate in una clausola possono infatti essere riutilizzate per
incrementare la dimensione di un'altra clausola, in quanto esse sono rese
obbligatoriamente false e non modificano il significato della WFF se riutilizzate.
Sia quindi $u_{\min}$ il numero di letterali presenti nella clausola con il minor numero di letterali all'interno della WFF $k$-SAT; è necessario definire solo $k - u_{\min}$ nuove
variabili per incrementare la dimensione di tutte le eventuali clausole con un numero di letterali $u < k$.\\
\\
Viene inoltre scelto di considerare solo WFF con clausole che \textit{non contengono} letterali ripetuti
o letterali opposti sulla stessa variabile: nel primo caso, infatti, esse possono essere semplificate, rimuovendo tutte le copie dello stesso letterale, mentre nel secondo caso le clausole sono tautologie, perchè qualsiasi sia il valore assunto dalla variabile $v$ su cui sono definiti i letterali opposti, la clausola risulta soddisfatta.\\
\\
Il numero di possibili clausole diverse che possono comparire in una WFF di un problema $k$-SAT
su $n$ variabili è quindi $m = 2^k \cdot \binom{n}{k}$, e si può di conseguenza rappresentare efficientemente la WFF come una stringa binaria di lunghezza $m$:
\begin{equation*}
    X = \ang{x_0, ..., x_{m-1}} \quad
    x_i = \begin{cases}
        0 & \tn{se clausola di indice } i \tn{ non compare nella WFF} \\
        1 & \tn{se clausola di indice } i \tn{ compare nella WFF}
    \end{cases}
\end{equation*}
La rappresentazione con stringa binaria viene definita efficiente perchè lo spazio
occupato, ovvero il numero di bit che la compongono, è $\Theta \bra{n^k}$ e, fissato $k$, cresce in maniera 
\textit{polinomiale} (proprio perchè il numero di possibili clausole cresce in 
maniera polinomiale).
Questa rappresentazione permette inoltre di evitare di memorizzare esplicitamente tutti i 
letterali che compaiono nella WFF: se si decidesse infatti di memorizzare la WFF in maniera esplicita, e sia $\log_2 2n$ il numero di bit necessari per rappresentare i $2n$ indici dei letterali, 
allora, nel caso peggiore in cui la WFF contenga
tutte le clausole, il numero di bit necessario sarebbe pari a 
$k \cdot \log_2 2n \cdot 2^k \cdot \binom{n}{k}$. Si avrebbe quindi ancora una crescita polinomiale, ma con 
una costante di crescita maggiore.\\
\\
Nell'effettiva implementazione, per migliorare il tempo di esecuzione, è stato scelto di definire una WFF 
non solo come una stringa binaria, ma come una struttura dati contenente:
\begin{itemize}
    \item la stringa binaria appena descritta;
    \item il tipo di problema $k$-SAT su cui è stata definita;
    \item il numero di variabili su cui è definita;
    \item la sua lunghezza, ovvero il numero di possibili clausole;
    \item il numero di clausole presenti;
    \item una struttura dati che memorizza il numero di volte che ogni letterale (sia 
    positivo che negativo) compare nella WFF.
\end{itemize}
Alla creazione della WFF, tutti questi attributi vengono calcolati e memorizzati 
nella struttura dati che la rappresenta, in modo tale da non dover rieseguire
più volte questi calcoli che sì, sono in tempo lineare o polinomiale, ma che 
nel caso di WFF di grandi dimensioni possono comunque richiedere molto
tempo.

\subsubsection{Rappresentare le clausole}
Sia $V$ l'insieme delle variabili su cui è definita la WFF.
A ogni variabile viene assegnato, per rappresentarla, un indice univoco $i \in \mathbb{N}$, $0 \le i < |V|$.\\
\\
Una clausola del problema $k$-SAT viene rappresentata semplicemente come:
\begin{align*}
    C &= \ang{c_0, \ldots, c_{k-1}} \quad c_i \in V\\
    B &= \ang{b_0, \ldots, b_{k-1}} \quad b_i \in \cbra{0, 1}
\end{align*}
ovvero come una $k$-upla di indici di variabili, a cui viene affiancata una $k$-upla di bit di negazione.
L'$i$-esimo letterale di una clausola è quindi definito sulla variabile $c_i$ ed è positivo se $b_i = 0$, negativo se $b_i = 1$.\\
\\
Si decide inoltre di definire una forma “standard” per le clausole, in modo da semplificarne
la manipolazione: i letterali sono quindi ordinati in ordine crescente in base 
all'indice della variabile su cui sono definiti, in modo tale che 
\begin{equation*}
    c_i < c_j \quad \forall i, j : 0 \le i < j < k
\end{equation*}
Esempio:\\
$\bra{\lnot 3, 16, 2, \lnot 8}$ non è in forma standard.\\
$\bra{2, \lnot 3, \lnot 8, 16}$ è in forma standard.\\
\\
Sulle clausole viene inoltre definita una relazione d'ordine e a ognuna di esse viene assegnato un indice, corrispondente alla posizione in cui compaiono nell'ordinamento.
Le clausole vengono quindi ordinate in ordine lessicografico crescente in base alla sequenza
di indici di variabili di cui sono composte e, se risultano uguali secondo questo criterio,
vengono ordinate in ordine lessicografico crescente in base alla sequenza di bit di negazione.\\
Di conseguenza, le $2^k$ varianti di uno stessa clausola compaiono consecutivamente all'interno della WFF.\\
\\
Data una generica $k$-clausola in forma standard di indice $i$ appartenente a una WFF definita su $n$ variabili, per passare alla clausola di indice $i+1$ si utilizza il seguente algoritmo:
\begin{algorithm}[H]
    \caption{Algoritmo per incrementare una clausola.}
    \label{alg:increment_clause}
    \begin{algorithmic}[1]
        \Procedure{Increment-clause}{clause, varNum}
            \For{j $\leftarrow$ clause.length - 1 to 0}
                \State clause.isNeg[j] $\gets$ !clause.isNeg[j]
                \If{clause.isNeg[j]}
                    \State \Return
                \EndIf
            \EndFor\\
            
            \For{i $\leftarrow$ clause.length - 1 to 0}
                \If{clause.vars[i] $<$ varNum - clause.length + i}
                    \State ++clause.vars[i]
                    \For{u $\leftarrow$ i to clause.length - 1}
                        \State clause[u] = clause[u - 1] + 1
                    \EndFor
                    \State \Return
                \EndIf
            \EndFor\\
            
            \State clause $\gets$ \Call{Construct-clause}{clause.length}
        \EndProcedure
    \end{algorithmic}
\end{algorithm}
Questo algoritmo restituisce solo clausole valide e permette di 
evitare di ottenere più volte la stessa clausola.\\
Non sarà in realtà necessario memorizzare esplicitamente le clausole che formano una WFF 
in quanto, utilizzando il loro indice, sarà possibile risalire a una clausola in tempo polinomiale.

\subsubsection{Rappresentare gli assegnamenti}
Gli assegnamenti su $n$ variabili vengono rappresentati invece come stringhe 
binarie di lunghezza $n$:
\[
    A = \ang{a_0, \ldots, a_{n-1}} \quad
    a_i = \begin{cases}
        0 & \tn{se la variabile di indice } i \tn{ assume valore } \textsc{false}\\
        1 & \tn{se la variabile di indice } i \tn{ assume valore } \textsc{true}
    \end{cases}
\]
Di conseguenza, dato un assegnamento, il valore booleano assunto da un 
generico letterale $l$ su una variabile di indice $i$ e con valore di bit di negazione $b$ è:
\begin{equation*}
    l = \begin{cases}
        \textsc{true} & (a_i = 1 \land b = 0) \lor (a_i = 0 \land b = 1) \\
        \textsc{false} & (a_i = 0 \land b = 0) \lor (a_i = 1 \land b = 1)
    \end{cases}
\end{equation*}
Può però tornare utile definire un assegnamento, che prende il nome di 
\textit{assegnamento parziale}, in cui 
alcune variabili non hanno ancora assunto un valore definitivo e possono
quindi considerarsi in una superposizione.
Per questo, la stringa binaria che rappresenta l'assegnamento viene affiancata da un'altra stringa
binaria di uguale lunghezza, la quale specifica quali variabili si trovano ancora in superposizione.
\[
    \begin{aligned}
        A &= \ang{a_0, \ldots, a_{n-1}}\\
        S &= \ang{s_0, \ldots, s_{n-1}}
    \end{aligned} \quad
    v_i = \begin{cases}
        \textnormal{superposizione} & s_i = 1 \\
        \textsc{true} & s_i = 0 \land a_i = 1 \\
        \textsc{false} & s_i = 0 \land a_i = 0
    \end{cases}
\]
Se un letterale è definito su una variabile in superposizione, esso potrebbe
essere considerato vero, ponendo un assegnamento alla variabile tale da renderlo vero.
Questo aspetto non è però utile negli algoritmi e complica la verifica di
molte clausole.
Per questo motivo, si è deciso arbitrariamente che, per verificare la soddisfacibilità di un letterale,
la variabile su cui è definito non deve trovarsi in superposizione.\\
\\
Vengono infine memorizzati, in ogni assegnamento, anche il numero di variabili su cui è definito
e il numero di variabili che si trovano in superposizione.

\clearpage

\section{Osservazioni di combinatoria}
\subsubsection{Numero di possibili clausole su n variabili}
Come già descritto nella sezione precedente, il numero di possibili clausole che è possibile definire in un problema $k$-SAT su $n$ variabili è pari a:
\begin{equation*}
    m = 2^k \cdot \binom{n}{k}
\end{equation*}
Bisogna infatti considerare tutti i possibili insiemi di $k$ elementi
che è possibile generare usando $n$ variabili, ovvero $\binom{n}{k}$.
Le $k$ variabili che compaiono all'interno di una clausola possono però
essere presenti nel loro letterale negativo, e tutte le possibili combinazioni
delle negazioni delle $k$ variabili sono pari a $2^k$.\\
Fissato un problema $k$-SAT, il numero di possibili clausole cresce
quindi in maniera polinomiale ($2^k$ è infatti una costante, dato $k$ costante), in particolare
$\Theta(n^k)$.

\begin{algorithm}
    \caption{Algoritmo per trovare il numero di possibili clausole su $n$ variabili}
    \label{alg:calc_clauses_num}
    \begin{algorithmic}
        \Procedure{Calc-clauses-num}{varNum, k}
            \State \Return \Call{Pow}{2, k} $\cdot$ \Call{Binom}{varNum, k} 
        \EndProcedure
    \end{algorithmic}
\end{algorithm}


\subsubsection{Numero di possibili WFF su n variabili}
Il numero di possibili WFF in un problema $k$-SAT su $n$ variabili è $2^m$,
dove $m$ è il numero di possibili clausole che è possibile definire su $n$ variabili:
al crescere del numero delle variabili (e quindi delle clausole), il numero 
di possibili WFF cresce quindi in maniera \textit{esponenziale}.
\begin{equation*}
    w = 2^m = 2^{2^k \cdot \binom{n}{k}}.
\end{equation*}

\begin{algorithm}[H]
    \caption{Algoritmo per trovare il numero di possibili WFF su $n$ variabili}
    \label{alg:calc_wff_num}
    \begin{algorithmic}
        \Procedure{Calc-wff-num}{varNum, k}
            \State clausesNum $\gets$ \Call{Calc-clauses-num}{varNum, k}
            \State \Return \Call{Pow}{2, clausesNum}
        \EndProcedure
    \end{algorithmic}
\end{algorithm}


\subsubsection{Numero di clausole soddisfatte da un assegnamento}
Ogni assegnamento verifica tutte le clausole, tra quelle possibili, che contengono almeno un letterale
che risulta vero. Il loro numero è ovviamente pari al numero di possibili 
clausole che è possibile definire, meno il numero di clausole non soddisfatte.
Calcolare il numero di clausole non soddisfatte è molto semplice: sono infatti
tutte quelle formate dai soli $n$ letterali che risultano falsi per l'assegnamento dato.\\
Il loro numero sarà quindi pari a:
\begin{equation*}
    f = \binom{n}{k}
\end{equation*}
Il numero di clausole soddisfatte è quindi:
\begin{equation*}
    t = 2^k \binom{n}{k} - \binom{n}{k} = \bra{2^k - 1} \binom{n}{k}
\end{equation*}
Di conseguenza, una WFF composta da un numero maggiore di clausole
sicuramente non ammette un assegnamento che le verifica tutte contemporaneamente e, quindi,
non è soddisfacibile.

\begin{algorithm}
    \caption{Algoritmo per trovare il limite superiore al numero di clausole in una WFF su $n$ variabili.}
    \label{alg:limit_sup_satisfiability}
    \begin{algorithmic}
        \Procedure{Upper-bound-satisfiability}{varNum, k}
            \State \Return (\Call{Pow}{2, k} - 1) $\cdot$ \Call{Binom}{varNum, k}
        \EndProcedure
    \end{algorithmic}
\end{algorithm}

\subsubsection{Numero di WFF soddisfatte da un assegnamento}
Le WFF soddisfatte da un assegnamento sono tutte quelle composte da sole 
clausole soddisfatte.
Ogni clausola soddisfatta può comparire o meno all'interno di una 
WFF e quindi, se $t$ è il numero di clausole soddisfatte dall'assegnamento,
il numero totale di WFF soddisfatte è pari a:
\begin{equation*}
    s = 2^t = 2^{\bra{2^k - 1} \cdot \binom{n}{k}}
\end{equation*}

\begin{algorithm}
    \caption{Algoritmo per trovare il numero di WFF soddisfatte da un assegnamento}
    \label{alg:count_sat_wffs}
    \begin{algorithmic}
        \Procedure{Count-satisfied-wffs}{varNum, k}
            \State satisfied-clauses $\gets$ \Call{Upper-bound-satisfiability}{varNum, k}
            \State \Return \Call{Pow}{2, satisfied-clauses}
        \EndProcedure
    \end{algorithmic}
\end{algorithm}


\subsubsection{Limite superiore al numero di WFF difficili}
Per un problema $k$-SAT su $n$ variabili, è stato detto che il numero di possibili clausole è pari
a $2^k \binom{n}{k}$.
Ovviamente, il numero di possibili WFF è $2^{2^k \binom{n}{k}}$.\\
Non tutte queste WFF sono però difficili da risolvere per un SAT solver, e una parte di esse può essere scartata in tempo polinomiale, dovendo quindi eseguire un eventuale algoritmo risolutivo solo su quelle rimaste.\\
\\
Per poter comprendere più facilmente quali WFF sono facili da risolvere, può essere utile ragionare in termini di clausole base e di loro varianti.\\
Una \textit{clausola base} è una clausola in cui tutti i letterali sono positivi,
mentre una \textit{variante} di una clausola di base è una clausola definita sulle stesse 
variabili ma in cui almeno un letterale è negativo. 
L'insieme contenente la clausola di base e le sue varianti sarà chiamato
\textit{insieme di varianti}. Ogni insieme di varianti, in un problema $k$-SAT su $n$ variabili, ha cardinalità $2^k$. Il numero di insiemi di varianti che è invece possibile definire è pari a $\binom{n}{k}$.\\
Un insieme di varianti non fa riferimento alle clausole contenute in una particolare WFF, ma a tutte le clausole che è possibile definire.\\
\\
Dato un assegnamento qualsiasi, le clausole di un singolo insieme di varianti completo non possono essere tutte contemporaneamente soddisfatte: in ognuno di essi, infatti, è sicuramente presente una singola clausola che non sarà soddisfatta dall'assegnamento.\\
Di conseguenza, qualsiasi WFF che presenta tutte le clausole di almeno un insieme di 
varianti è insoddisfacibile. La presenza di un insieme di varianti completo in una WFF può
essere decisa in tempo polinomiale $O\bra{n^k}$.\\
In questo caso, oltre agli insiemi di varianti, può essere utile definire una stringa 
binaria di lunghezza $\binom{n}{k}$ in cui l'$i$-esimo bit è 1
se l'$i$-esimo insieme di varianti (o meglio le sue clausole) è presente nella WFF presa 
in considerazione.
Se almeno uno di questi bit è 1, allora la WFF è insoddisfacibile.\\
Il numero di WFF che presentano almeno un insieme di varianti completo (e sono quindi 
insoddisfacibili) può essere calcolato sottraendo il numero di WFF che non 
presentano alcun insieme di varianti completo al numero totale 
di WFF che è possibile definire.\\
Il numero di WFF formate dalle sole clausole di un singolo insieme di varianti è pari a $2^{(2^k)}$. La WFF contenente tutte le clausole presenta però l'insieme di varianti completo, e non può quindi essere accettata per un'eventuale soddisfacibilità: di conseguenza, le possibili WFF sono $2^{(2^k)} - 1$.
Essendo il numero di insiemi di varianti pari a $\binom{n}{k}$, il numero di WFF che non presentano alcun insieme di varianti completo, e quindi un limite 
superiore al numero di WFF difficili, è pari a:
\begin{equation*}
    \bra{2^{\bra{2^k}} - 1}^{\binom{n}{k}}
\end{equation*}
Il numero di WFF che presentano almeno un insieme di varianti completo, e sono quindi 
insoddisfacibili, è invece pari a:
\begin{equation*}
    2^{2^k \binom{n}{k}} - \bra{2^{\bra{2^k}} - 1}^{\binom{n}{k}}
\end{equation*}
Questo limite superiore può essere ulteriormente abbassato considerando le WFF composte da sole clausole di Horn, risolvibili in tempo polinomiale.
Una clausola di Horn è una disgiunzione di $k$ letterali, di cui al massimo uno è positivo.
Si può osservare che un insieme di varianti, definito sulle variabili $v_0, \ldots, v_{k-1}$ contiene esattamente $k+1$ clausole di Horn:
\begin{itemize}
    \item $(v_0,$ ¬$v_1, \ldots,$ ¬$v_{k-1})$
    \item (¬$v_0, v_1, \ldots,$ ¬$v_{k-1})$
    \item $\ldots$
    \item (¬$v_0,$ ¬$v_1, \ldots, v_{k-1})$
    \item (¬$v_0,$ ¬$v_1, \ldots,$ ¬$v_{k-1})$
\end{itemize}
Di conseguenza, una WFF composta da sole clausole di Horn non presenta insiemi di varianti completi.\\
Il numero di WFF con sole clausole di Horn che è possibile definire su $n$ variabili in un problema $k$-SAT è pari a:
\begin{equation*}
    \bra{2^{\bra{k+1}}}^{\binom{n}{k}} = 2^{\bra{k+1} \binom{n}{k}}
\end{equation*}
Il nuovo limite superiore alle WFF difficili è:
\begin{equation*}
    \bra{2^{\bra{2^k}} - 1}^{\binom{n}{k}} - 2^{\bra{k + 1} \binom{n}{k}}
\end{equation*}
In realtà, tale limite può essere ulteriormente abbassato applicando lo stesso 
ragionamento anche per altre classi di WFF trattabili, come quelle legate al problema
della piccionaia, che devono però poter essere individuabili in tempo trattabile.
Se infatti una WFF, anche facile da risolvere, richiedesse tempo esponenziale per 
individuare la classe di WFF a cui appartiene, allora questa procedura non porterebbe
alcun vantaggio e avrebbe più senso eseguire direttamente il SAT solver sulla WFF.\\
\\
Si può notare, inoltre, che il limite superiore al numero di clausole presenti in una WFF,
pari a $(2^k - 1) \binom{n}{k}$, è in realtà un caso particolare di quanto appena
spiegato: superare questo 
numero di clausole implica la presenza di un insieme di varianti completo all'interno di 
essa e quindi la sua insoddisfacibilità.\\
\\
Gli insiemi di varianti possono anche essere utilizzati in un algoritmo
risolutivo per il problema $k$-SAT per individuare in tempo polinomiale eventuali
assegnamenti obbligatori su certe variabili.
Sia infatti $M$ un insieme di varianti definito su $k$ variabili, e 
siano scelte $u$ variabili tra queste. Se l'insieme di varianti contiene le $2^u$ clausole
che presentano tutte le varianti delle $u$ variabili e condividono i rimanenti $k-u$ 
letterali, allora queste possono essere rimosse e sostituite con un'unica clausola 
contenente solo gli altri $k-u$ letterali.\\
La presenza di queste clausole indica infatti che, qualunque sia il valore assunto dalle
$u$ variabili, almeno uno degli altri $k-u$ letterali deve essere vero.
\begin{exmp}
    \[\tn{WFF: } (x_1 \lor x_2 \lor x_3 \lor x_4) \land (\tn{¬}x_1 \lor x_2 \lor x_3 \lor x_4) \land (x_1 \lor x_2 \lor \tn{¬}x_3 \lor x_4) \land (\tn{¬}x_1 \lor x_2 \lor \tn{¬}x_3 \lor x_4)\]
    \begin{align*}
        (x_1 \lor x_2 \lor x_3 \lor x_4) &\implies (\tn{¬}x_1 \land \tn{¬}x_3) \rightarrow (x_2 \lor x_4)\\
        (\tn{¬}x_1 \lor x_2 \lor x_3 \lor x_4) &\implies (x_1 \land \tn{¬}x_3) \rightarrow (x_2 \lor x_4)\\
        (x_1 \lor x_2 \lor \tn{¬}x_3 \lor x_4) &\implies (\tn{¬}x_1 \land x_3) \rightarrow (x_2 \lor x_4)\\
        (\tn{¬}x_1 \lor x_2 \lor \tn{¬}x_3 \lor x_4) &\implies (x_1 \land x_3) \rightarrow (x_2 \lor x_4)
    \end{align*}
\end{exmp}
Essendo la cardinalità di un insieme di varianti costante al crescere del numero di variabili, individuare questo caso particolare richiede tempo costante $O(1)$.
Essendo il numero di insiemi di varianti pari a $\binom{n}{k}$, il tempo richiesto 
per analizzare tutti gli insiemi di varianti per una WFF è pari a:
\begin{equation*}
    \Theta \bra{\binom{n}{k}} = \Theta \bra{n^k}
\end{equation*}
Questo approccio non trova in realtà grande applicazione nel lavoro portato avanti nei
successivi capitoli: come già descritto, la WFF mantiene solo $k$-clausole, ed eventuali
clausole di dimensione minore ottenute con questo ragionamento vengono nuovamente ingrandite in modo tale da avere esattamente $k$ 
letterali.
Può essere applicato però se $u = k-1$: in questo caso, infatti, la presenza delle $2^u$ clausole della forma precedentemente descritta implica obbligatoriamente che l'unico letterale rimanente (e condiviso tra le clausole) sia vero e che esse possano essere rimosse dalla WFF.
Di conseguenza, la variabile su cui il letterale è definito assume un valore definitivo, che non potrà essere cambiato nemmeno con backtracking. Se si verifica un conflitto su questa variabile durante l'esecuzione del SAT solver, allora la WFF è sicuramente non soddisfacibile.