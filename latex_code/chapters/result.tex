\chapter{Conclusioni}
Come riscontrato nell'implementazione ed esecuzione degli algoritmi di risoluzione, si è notato come il problema SAT sia in realtà trattabile in molti casi; si sono però individuati i casi particolari in cui la sua risoluzione si complica notevolmente, come nel caso delle WFF il cui rapporto tra il numero di clausole e il numero di variabili si trova nell'intorno della transizione di fase o nel caso delle formule che codificano il problema della piccionaia senza clausole di definizione.\\
Lo studio della struttura delle WFF ha inoltre permesso una più chiara comprensione di quali e quante siano effettivamente le WFF per cui è utile utilizzare un SAT solver, in modo tale da evitare l'applicazione di algoritmi con tempi esponenziali nel caso peggiore per WFF di cui è nota la facilità di risoluzione.\\
Alcuni aspetti possono però essere ulteriormente migliorati, come il limite superiore al numero di WFF difficili da risolvere, che può essere abbassato, o la scoperta di ulteriori casi di insiemi di varianti che portano all'insoddisfacibilità.\\
Per quanto riguarda la rappresentazione binaria delle WFF adottata, si sono individuati sia dei vantaggi, come una rappresentazione più strutturata che permette una più facile analisi e, in confronto alle rappresentazioni più tradizionali, una minor quantità di memoria richiesta quando la WFF è formata da molte clausole, ma anche degli svantaggi, come uno spreco di memoria per WFF con poche clausole o un tempo di esecuzione leggermente maggiore rispetto alle altre rappresentazioni, anche se ancora polinomiale nel numero di variabili.