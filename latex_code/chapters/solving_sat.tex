\chapter{Risoluzione del problema k-SAT}

Per stabilire se una WFF è soddisfacibile si potrebbe
pensare di testare semplicemente tutti i possibili assegnamenti. 
Il numero di possibili assegnamenti è però $2^n$, con $n$ numero di variabili su cui è definita la WFF, 
e quindi cresce in maniera \textit{esponenziale}.
L'obiettivo delle prossime sezioni è quindi quello di trovare
casi particolari in cui non è necessario testare alcun assegnamento per 
stabilire la soddisfacibilità di una WFF o, nel caso in cui sia necessario,
ridurre il più possibile l'utilizzo di assegnamenti che non verificano la WFF
per testare la sua soddisfacibilità.

\section{Partizionamento dell'insieme delle clausole}

Un possibile approccio a una migliore verifica della soddisfacibilità
è il partizionamento dell'insieme delle clausole che formano una WFF.\\
Si può definire una partizione formata da $n$ sottoinsiemi a cui si associa
un indice $i$, $0 \le i < n$, che identifica
la variabile su cui un sottoinsieme è definito, e un bit di negazione.
Siano le variabili $x_0, \ldots, x_{n-1}$ e i bit $b_0, \ldots, b_{n-1}$,
al sottoinsieme di indice $i$ corrisponderà il letterale $x_i$ se $b_i = 0$, $\lnot x_i$ se $b_i = 1$.
Ogni sottoinsieme di indice $i$ contiene clausole che contengono
tutte il letterale associato al sottoinsieme.\\
\\
Una WFF è soddisfacibile se e solo se è possibile trovare un assegnamento ai bit $b_0, \ldots, b_{n-1}$ tale che è possibile partizionare l'insieme delle clausole che la compongono.
Una volta partizionato l'insieme delle clausole, è facile trovare uno 
degli assegnamenti che verificano la WFF: se il sottoinsieme di indice $i$ contiene clausole con il letterale $x_i$, allora
la variabile $x_i$ assume valore true nell'assegnamento; se invece
il sottoinsieme contiene clausole con il letterale 
$\lnot x_i$, allora la variabile $x_i$ assume valore false.\\
Siccome il numero di possibili partizionamenti è pari a $2^n$, questo approccio
rimane esponenziale.

\begin{exmp}
    WFF: $(x_3 \lor \lnot x_2 \lor x_5) \land (x_1 \lor x_3 \lor x_5) \land (\lnot x_4 \lor x_3 \lor x_5) \land (x_3 \lor x_4 \lor x_5)$\\
    \[
        \underbrace{(x_1 \lor x_3 \lor x_5)}_{x_1} \land 
        \underbrace{(\lnot x_2 \lor x_3 \lor x_5)}_{\lnot x_2} \land 
        \underbrace{(x_3 \lor \lnot x_4 \lor x_5) \land (x_3 \lor x_4 \lor x_5)}_{x_3}
    \]
    In questo caso le clausole sono state partizionate in tre sottoinsiemi, il 
    primo di clausole contenenti tutte il letterale positivo $x_1$, il secondo 
    con clausole contenenti il letterale negativo $\lnot x_2$ e il terzo 
    contenenti il letterale positivo $x_3$.\\
    Un assegnamento che verifica la WFF è quindi 
    \[
        x_1 = \textsc{true} \, , x_2 = \textsc{false} \, , x_3 = \textsc{true} \, , x_4=\ldots \, , x_5=\ldots
    \]
\end{exmp}

\section{DPLL su WFF rappresentate come stringhe binarie}
L'implementazione dell'algoritmo DPLL su WFF come stringhe binarie segue tutte le 
indicazioni date nelle sezioni 2.4.3 e 2.4.4, ma non è immediata come si potrebbe pensare ed è 
necessario effettuare alcune modifiche.
La prima, e la più importante, è quella che riguarda la dimensione delle clausole della 
WFF durante l'esecuzione dell'algoritmo: DPLL basa il suo funzionamento su WFF con 
clausole con \textit{al più} $k$ letterali e sulla rimozione dei letterali falsi da 
esse, che di fatto porta a un loro accorciamento.
La rappresentazione binaria definita precedentemente ammette però solo WFF con esattamente $k$ letterali 
per clausola.
Possono essere quindi adottati due approcci per l'implementazione di DPLL:
\begin{itemize}
    \item detto $k$ il numero di letterali contenuti in ogni clausola della WFF fornita in
    input all'algoritmo, vengono mantenute $k-1$ WFF aggiuntive su clausole di dimensione
    $u, \forall u, 1 \le u < k$.\\
    Quando viene rimosso un letterale da una $u$-clausola, allora essa si riduce di 
    dimensione: viene quindi rimossa dalla $u$-WFF e viene aggiunta la corrispondente clausola
    su $u-1$ letterali alla $(u-1)$-WFF;
    \item viene mantenuta solo la WFF passata in input all'algoritmo.
    L'assegnamento a una variabile porta alla rimozione di tutte le clausole che
    contengono il letterale sulla variabile soddisfatto, mentre il letterale sulla
    variabile non soddisfatto non viene rimosso dalle clausole, ma l'algoritmo è a
    conoscenza della sua falsità.
\end{itemize}
Entrambi questi approcci hanno dei vantaggi e degli svantaggi.
Il primo porta a un'esecuzione rapida (alla pari degli altri SAT solver basati su DPLL),
ma richiede una quantità di memoria maggiore: devono essere infatti memorizzate $k$ WFF.
La gran parte dei bit utilizzati per la loro rappresentazione, inoltre, è impostata a 0, e
non porta quindi con sè informazioni significative sulle clausole della WFF che si sta
trattando.\\
Sia $k$ la dimensione delle clausole della WFF passata in input all'algoritmo, con $k \ge
1$; la quantità di memoria, considerando solo le stringhe binarie, necessaria per
affrontare il problema usando il primo approccio è pari a:
\begin{center}
    $\sum_{i=1}^{k} 2^i \binom{n}{i}$
\end{center}
Il secondo approccio permette di risparmiare memoria, ma l'algoritmo deve essere in grado
di tenere traccia o rilevare quali letterali di una clausola ancora presente sono
effettivamente definiti su variabili non ancora assegnate e quali sono falsi.
Questo allunga il tempo di esecuzione: ogni qualvolta si cerca una variabile di splitting
o si effettua la unit propagation devono essere effettuati dei controlli sui
letterali delle clausole che, nel caso peggiore, possono anche richiedere di scorrere
$k-1$ letterali già falsi prima di ottenere un letterale su una variabile non ancora
assegnata.\\
\\
Dovendo utilizzare, per le esecuzioni, un computer con poca memoria principale, si è deciso di implementare l'algoritmo DPLL utilizzando il secondo approccio, ovvero quello che mantiene una WFF con esattamente $k$ letterali.
Non è comunque difficile applicare gli stessi ragionamenti nello sviluppo di un algoritmo DPLL basato sul primo approccio.

\begin{algorithm}[H]
    \caption{Algoritmo DPLL per la verifica della soddisfacibilità di una WFF che mantiene una WFF con esattamente $k$ letterali per clausola.}
    \label{alg:dpll_second_approach}
    \begin{algorithmic}[1]
        \Procedure{DPLL}{wff, partialAssg}
            \If{wff.varNum $\neq$ partialAssg.length}
                \State \textbf{error}
            \EndIf\\
        
            \If{\Call{Has-pure-lit}{wff, partialAssg}}
                \State \Return \Call{Pure-lit-heuristic}{wff, partialAssg}
            \EndIf
            \If{\Call{Has-unit-clause}{wff, partialAssg}}
                \State \Return \Call{Unit-propagation}{wff, partialAssg}
            \EndIf\\

            \State splitVar $\gets$ \Call{Find-most-present-var}{wff, partialAssg}
            \State tmpWff $\gets$ wff
            \State partialAssg.varValues[splitVar] $\gets$ 0
            \State partialAssg.inSuperpos[splitVar] $\gets$ 0
            \State \Call{Update-wff}{tmpWff, partialAssg}
            \If{\Call{DPLL}{tmpWff, partialAssg}}
                
                \State result $\gets$ true
            \Else
                \State tmpWff $\gets$ wff
                \State partialAssg.varValues[splitVar] $\gets$ 1
                \State \Call{Update-wff}{tmpWff, partialAssg}
                \State result $\gets$ \Call{DPLL}{tmpWff, partialAssg}
            \EndIf
            \State partialAssg.inSuperpos[splitVar] $\gets$ 1
            \State \Return result
        \EndProcedure
    \end{algorithmic}
\end{algorithm}

\begin{algorithm}[H]
    \caption{Algoritmo di unit propagation per DPLL con esattamente $k$ letterali per clausola.}
    \label{alg:unit_propagation_second_approach}
    \begin{algorithmic}[1]
        \Procedure{Unit-propagation}{wff, partialAssg}
            \If{wff.varNum $\neq$ partialAssg.length}
                \State \textbf{error}
            \EndIf\\
            
            \State var $\gets$ NULL
            \State tmpCl $\gets$ \Call{Construct-clause}{wff.kSat}
            \For{i $\gets$ 0 to wff.length - 1}
                \If{wff.isClPresent[i]}
                    \State unassignedLits $\gets$ 0
                    \State unassignedVar $\gets$ NULL
                    \State boolToAssign $\gets$ NULL
                    \For{j $\gets$ 0 to clause.length - 1}
                        \If{partialAssg.inSuperpos[clause.vars[j]]}
                            \State ++unassignedLits
                            \State unassignedVar $\gets$ clause.vars[j]
                            \State boolToAssign $\gets$ !clause.isNeg[j]
                        \EndIf
                    \EndFor\\
                    
                    \If{unassignedLits = 1}
                        \State partialAssg.varValues[unassignedVar] $\gets$ boolToAssign
                        \State partialAssg.inSuperpos[unassignedVar] $\gets$ 0
                        \State tmpWff $\gets$ wff
                        \State \Call{Update-wff}{tmpWff, partialAssg}
                        \State result $\gets$ \Call{DPLL}{tmpWff, partialAssg}
                        \State partialAssg.inSuperpos[unassignedVar] $\gets$ 1
                        \State \Return result
                    \EndIf
                \EndIf
                \State \Call{Increment-clause}{tmpCl, wff.varNum}
            \EndFor
            \State \textbf{error}
        \EndProcedure
    \end{algorithmic}
\end{algorithm}

\begin{algorithm}[H]
    \caption{Algoritmo per l'euristica del letterale puro su WFF con esattamente $k$ letterali per clausola.}
    \label{alg:pure_literal_second_approach}
    \begin{algorithmic}[1]
        \Procedure{Pure-lit-heuristic}{wff, partialAssg}
            \If{wff.varNum $\neq$ partialAssg.length}
                \State \textbf{error}
            \EndIf\\

            \For{i $\gets$ 0 to wff.varNum - 1}
                \If{partialAssg.inSuperpos[i]}
                    \If{wff.litCount[i] = 0}
                        \State partialAssg.inSuperpos[i] $\gets$ 0
                        \State partialAssg.varValues[i] $\gets$ false
                    \ElsIf{wff.litCount[i + wff.varNum] = 0}
                        \State partialAssg.inSuperpos[i] $\gets$ 0
                        \State partialAssg.varValues[i] $\gets$ true
                    \EndIf\\

                    \If{!partialAssg.inSuperpos[i]}
                        \State tmpWff $\gets$ wff
                        \State \Call{Update-wff}{tmpWff, partialAssg}
                        \State result $\gets$ \Call{DPLL}{tmpWff, partialAssg}
                        \State partialAssg.inSuperpos[i] $\gets$ 1
                        \State \Return result
                    \EndIf
                \EndIf
            \EndFor
            \State \textbf{error}
        \EndProcedure
    \end{algorithmic}
\end{algorithm}


\subsection{Scegliere la variabile di splitting}
L'algoritmo DPLL richiede, durante la sua esecuzione, di poter individuare una variabile non ancora assegnata, in modo tale da assegnarle un valore.
La scelta della variabile non influisce sull'individuazione di un'eventuale soddisfacibilità della WFF: se essa è soddisfacibile, allora l'algoritmo DPLL giungerà al risultato corretto, indipendentemente dalla variabile di splitting scelta a un certo passo di esecuzione.
Una scelta ragionata può però portare a una diminuzione dei tempi di esecuzione: deve quindi essere individuato un criterio con cui effettuarla. Non esiste un criterio dimostrato con cui farlo e, perciò, vengono usate delle euristiche per selezionare quella che, secondo certi criteri assunti validi, porta idealmente al minor tempo di esecuzione.\\
\\
Le euristiche più utilizzate sono:
\begin{itemize}
    \item unit propagation: se è presente una unit clause, la variabile scelta è quella su cui è definito l'unico letterale presente nella clausola ed essa assume un valore tale che il letterale sia vero.
    Inoltre, non si necessita di testare l'assegnamento opposto sulla variabile nel caso in cui quello corrente porti all'insoddisfacibilità: se si verifica questo caso, allora viene semplicemente restituito un risultato negativo al chiamante;
    \item pure literal heuristic: viene scelta una variabile che compare solo come letterale positivo o solo come letterale negativo all'interno della WFF.
    Questa euristica è utile perchè, applicandola, è possibile rimuovere tutte le clausole che contengono il letterale puro sulla variabile scelta.
    Se si arriva all'insoddisfacibilità, inoltre, non è necessario testare l'assegnamento opposto sulla variabile.
    L'utilizzo di questa euristica permette di ridurre lo studio della soddisfacibilità di una WFF su $n$ variabili a una su $(n-1)$ variabili, in quanto la variabile di splitting viene completamente rimossa: nel caso di rappresentazione binaria, quindi, la quantità di memoria richiesta per rappresentare la WFF diminuisce;
    \item most present variable: viene scelta la variabile che compare in più clausole all'interno della WFF. Essa, trovandosi in molte clausole, pone vincoli su molte altri variabili e, di conseguenza, il suo assegnamento si ripercuote notevolmente sui valori che le altre variabili possono assumere.\\
    Se una variabile è presente in tutte le clausole, sia come letterale positivo che come letterale negativo, allora la soddisfacibilità della WFF $k$-SAT su $n$ variabili può essere studiata su due WFF $(k-1)$-SAT di $(n-1)$-variabili: l'assegnamento alla variabile porta infatti, oltre che a rimuovere tutte le clausole in cui è contenuto il suo letterale che risulta soddisfatto, anche a rimuovere il suo letterale falso da tutte le altre clausole. Di conseguenza, tutte le clausole rimanenti vengono diminuite di dimensione, e la variabile di splitting usata sicuramente non comparirà in esse.
\end{itemize}

\subsection{Possibili miglioramenti}

\subsubsection{Multithreading}
L'esecuzione parallela di diversi flussi indipendenti può portare a notevoli miglioramenti
ai tempi di esecuzione ma, in alcuni casi, può anche peggiorare le performance:
bisogna quindi individuare quelle situazioni critiche da gestire con più attenzione.\\
\\
La prima osservazione che è possibile fare è sulle dimensioni della WFF, ovvero
il suo numero di variabili e su quale problema $k$-SAT è definita.
Una WFF di piccole dimensioni non necessita di multithreading: il costo di creazione dei thread
e del loro context switch vanifica infatti tutti i miglioramenti, peggiorando i tempi di esecuzione.\\
\\
La seconda osservazione può essere fatta sul numero di thread utilizzati.
Si potrebbe infatti pensare di generare un thread per ogni flusso di esecuzione indipendente,
pensando che essi vengano effettivamente eseguiti tutti contemporanemante.\\
Per avere però un'idea di ciò a cui si va effettivamente incontro bisogna discutere di come una CPU multi-core
funziona e come il sistema operativo gestisce il multitasking.
Considerando un'architettura con una singola CPU (il discorso si può estendere facilmente anche
alle archittetture multi-CPU), la CPU può essere costituita da uno o più core, unità
di esecuzione fisiche indipendenti che possono eseguire contemporanemante thread differenti.
Ogni volta che un thread finisce di usare un core, un nuovo thread, scelto dal sistema operativo
tramite una politica di scheduling, si approria della risorsa appena rilasciata. Prima
che il nuovo thread venga effettivamente eseguito dal core, però, il sistema operativo deve effettuare
un context switch, ovvero deve aggiornare lo stato del core in modo tale che l'esecuzione possa avvenire
correttamente. Il context switch ha però, anche se minimo, un overhead, di cui bisogna tenere conto.\\
Tornando quindi al numero di thread che il programma utilizza per testare tutti i possibili assegnamenti, 
creare molti thread può portare a un grosso bottleneck, causato dai continui context switch che si 
verificano per l'assegnamento dei core, e quindi all'aumento dei tempi di esecuzione.
Per questo motivo, il numero di thread dovrebbe essere circa il doppio rispetto al numero
di core presenti, in modo da sfruttare a pieno la CPU senza però cadere nel caso appena descritto.\\
\\
Facendo riferimento all'albero degli assegnamenti esplorato dall'algoritmo DPLL, l'esplorazione dell'albero senza usare il multithreading 
avviene in profondità, ovvero viene esplorato completamente un ramo dalla radice a una foglia prima di
passare al ramo successivo. Sfruttando il multithreading, invece, $2 \cdot \textsc{numero di core}$
flussi di esecuzione possono essere eseguiti contemporanemante.
Bisogna quindi trovare il punto giusto dell'albero degli assegnamenti in cui inizializzare questi thread, 
in modo tale da massimizzare l'utilizzo della CPU.\\
Si è scelto di iniziare a creare i thread immediatamente dal nodo radice, terminando alla profondità
$\log_2 (2 \cdot \textsc{numero di core})$ esclusa, se e solo se l'altezza totale dell'albero, ovvero il numero 
di variabili, è maggiore di quest'ultima.
In questo modo, WFF su poche variabili verranno testate sequenzialmente, evitando così l'overhead prodotto dalla creazione
dei thread e dal context switch, mentre quelle su molte variabili godranno dei vantaggi del multithreading.\\
\\
Utilizzando questo approccio, però, la scoperta di un assegnamento valido in un thread non ha alcuna
conseguenza sull'esecuzione degli altri thread, che continuano a decidere la soddisfacibilità della WFF.
Si può però definire una variabile booleana condivisa che, quando impostata a true
da un thread che ha trovato un assegnamento valido, segnala a tutti gli altri di interrompere
la loro esecuzione restituendo il valore true, in quanto non più necessaria.
Gli altri thread verificano il valore della variabile all'inizio di ogni chiamata ricorsiva, ovvero
a ogni passaggio da un nodo di profondità $p$ a uno di profondità $p+1$ nell'albero degli assegnamenti.\\
Non è necessario, inoltre, utilizzare alcuna tecnica di sincronizzazione tra thread: le operazioni di lettura e scrittura su una variabile di tipo booleano, infatti, sono operazioni atomiche, e
non si presenta quindi alcun rischio di loro sovrapposizione. Ciò che però bisogna effettivamente analizzare, per poter affermare l'inutilità di tecniche di sincronizzazione, è l'ordine di esecuzione delle operazioni. \\
Qualsiasi sia la sequenza di operazioni di lettura o scrittura che deve essere effettuata sulla variabile, allora si può affermare che, al termine dell'esecuzione di tutte le operazioni, la variabile conterrà il valore true: ogni thread può infatti scrivere solo questo valore al suo interno, e non può ripristinare il suo stato originale, ovvero il valore false.
Un altro aspetto critico da considerare sono le scritture su risorse condivise condizionate dallo stato della variabile (e quindi dalla sua lettura): in questo caso, infatti, un altro thread potrebbe modificare lo stato della variabile dopo la lettura ma prima della scrittura, rendendo quindi non valido il dato precedentemente letto e di conseguenza il ramo di esecuzione che si sta portando avanti.
Non sono però presenti riscritture condizionate, e questo problema viene completamente evitato.\\
Si può però presentare un piccolo inconveniente: un thread potrebbe infatti leggere lo stato della variabile prima che un altro, pronto ad aggiornarla, possa effettivamente segnalare l'individuazione di un assegnamento valido. Questo porta, quindi, a un'ulteriore esecuzione del thread lettore, che deve aspettare un'ulteriore chiamata ricorsiva prima di poter effettivamente rilevare il risultato positivo dell'altro thread. La semantica del programma non viene però modificata. Questo svantaggio è compensato dal fatto di non dover utilizzare tecniche di sincronizzazione, e di non dover quindi preoccuparsi della difficoltà di gestione e l'overhead introdotto da esse.

\subsubsection{Watched literals}
Quando viene effettuato un assegnamento a una variabile, tutte le clausole che contengono
il letterale sulla variabile soddisfatto devono essere rimosse, mentre quelle che contengono il letterale non soddisfatto devono essere accorciate, rimuovendo quest'ultimo.
Questo però, in un algoritmo non ottimizzato, implica di dover scorrere più volte tutta la WFF per individuare quali clausole contengano effettivamente la variabile.\\
Ha quindi senso utilizzare, per ogni variabile, due liste di indici di posizione all'interno della WFF: la prima conterrà tutti gli indici delle clausole che contengono il letterale positivo sulla variabile, mentre la seconda quelli di clausole che contengono il letterale negativo.
Questa tecnica prende il nome di \textit{watched literals}, e permette di trattare WFF con moltissime clausole, addirittura nell'ordine dei milioni.\\
Quando viene effettuato un assegnamento su una variabile, viene scorsa la lista del letterale soddisfatto e tutte le clausole puntate dagli indici contenuti vengono rimosse.
L'altra lista viene invece usata per trovare immediatamente le clausole da cui bisogna rimuovere il letterale non soddisfatto, anche se nella precedente implementazione questo non viene fatto perchè i letterali falsi vengono mantenuti nella WFF.
Ciò permette inoltre di evitare di dover stabilire quali tra le possibili $2^k \binom{n}{k}$ clausole sono effettivamente presenti: si procede infatti all'accesso diretto a clausole di cui è garantita la presenza, e non è quindi necessario verificare che i relativi bit della WFF siano impostati a 1.

\subsubsection{Rappresentazione di WFF, clausole e assegnamenti come interi}
Nelle precedenti sezioni e algoritmi, le WFF, le clausole e gli assegnamenti sono stati rappresentati e manipolati come vettori di bit o valori booleani, e quindi gestiti con le usuali operazioni dei vettori. Ciò ha permesso un'implementazione più semplice degli algoritmi e ha facilitato lo studio della struttura delle WFF.\\
Può risultare però più performante rappresentare questi oggetti come interi. 
La quantità di memoria utilizzata non diminuisce, ma la loro manipolazione può essere velocizzata: le operazioni binarie, di shift e di somma/sottrazione di potenze di 2 sono infatti molto rapide, e permettono di evitare, in alcuni casi, di dover scorrere l'intero array di bit, come avviene invece nel caso dell'incremento delle clausole.

\section{Ricerca locale su WFF rappresentate come stringhe binarie}
L'implementazione di un algoritmo di ricerca locale su WFF rappresentate come stringhe binarie segue
tutte le osservazioni della sezione 1.4.5.
Si ricorda, però, che questo tipo di algoritmi non possono essere considerati completi: se essi trovano un assegnamento che porta alla soddisfacibilità, allora il risultato è corretto; se però non riescono a trovare alcun assegnamento, non è detto che la WFF sia insoddisfacibile, in quanto l'algoritmo può essersi assestato su un minimo locale, nonostante sia presente un minimo globale.

\begin{algorithm}[H]
    \caption{Algoritmo di ricerca locale per la risoluzione del problema SAT.}
    \label{alg:local_search}
    \begin{algorithmic}[1]
        \Procedure{Local-search}{wff, maxTries, maxFlips}
            \For{i $\gets$ 0 to maxTries - 1}
            \State tmpAssg $\gets$ \Call{Generate-random-assignment}{wff.varNum}
                \For{j $\gets$ 0 to maxFlips - 1}
                    \If{\Call{Is-wff-satisfied}{wff, tmpAssg}}
                        \State \Return true
                    \EndIf
                    \State bestNeighbour $\gets$ \Call{Find-best-neighbour}{wff, tmpAssg}
                    \If{tmpAssg = bestNeighbour}
                        \Comment{Minimo locale}
                        \State \textbf{break}
                    \EndIf
                    \State tmpAssg $\gets$ bestNeighbour
                \EndFor
            \EndFor
            \State \Return false
        \EndProcedure
    \end{algorithmic}
\end{algorithm}

\begin{algorithm}[H]
    \caption{Algoritmo per trovare il migliore assegnamento neighbour.}
    \label{alg:find_best_neighbour}
    \begin{algorithmic}[1]
        \Procedure{Find-best-neighbour}{wff, assg}
            \If{wff.varNum $\neq$ assg.length}
                \State \textbf{error}
            \EndIf\\
            
            \State bestNeighbour $\gets$ assg
            \State unsatClausesNum $\gets$ \Call{Count-unsat-clauses}{wff, assg}
            \State tmpAssg $\gets$ assg
            \For{i $\gets$ 0 to assg.length - 1}
                \State tmpAssg.varValues[i] $\gets$ !tmpAssg.varValues[i]
                \State tmpCount $\gets$ \Call{Count-unsat-clauses}{wff, assg}
                \If{tmpCount $<$ unsatClausesNum}
                    \State bestNeighbour $\gets$ tmpAssg
                    \State unsatClausesNum $\gets$ tmpCount
                \EndIf
                \State tmpAssg.varValues[i] $\gets$ !tmpAssg.varValues[i]
            \EndFor
            \State \Return bestNeighbour
        \EndProcedure
    \end{algorithmic}
\end{algorithm}
Implementazioni più moderne di un algoritmo di ricerca locale per il problema SAT utilizzano, al posto della funzione di conteggio del numero di clausole non soddisfatte, 
distribuzioni di probabilità sulle variabili: ogni volta che viene effettuata l'analisi dei neighbour, la loro distribuzione viene aggiornata e viene effettuato il flip della variabile che presenta la probabilità minore o maggiore, in base al criterio adottato.
Questo approccio permette di evitare l'analisi di tutti gli assegnamenti neighbour e di raggiungere più velocemente un eventuale minimo globale, riducendo quindi il tempo di esecuzione dell'algoritmo.

\section{Sottoformule indipendenti}
Le clausole di una WFF non fanno altro che specificare dei vincoli sui valori che le
variabili possono assumere: l'assegnamento di un valore booleano a una certa variabile, infatti, si ripercuote sui valori che le altre  possono assumere e in alcuni casi, come visto nel caso della unit propagation, può portare ad assegnamenti a cascata.
Non sempre però un assegnamento a una variabile pone dei vincoli su tutte le altre $n-1$ variabili, ma solo su un sottoinsieme di esse. Possono quindi essere presenti, nella WFF, delle sottoformule definite su sottoinsiemi completamente indipendenti di variabili, analizzabili separatamente da un SAT solver.\\
\\
Esempio:\\
$(x_5 \lor$ ¬$x_6 \lor $¬$x_7) \land ($¬$x_1 \lor $¬$x_2 \lor x_3) \land ($¬$x_1, $¬$x_2, $¬$x_3) \land (x_4 \lor x_5 \lor x_7)$\\
presenta due sottoformule indipendenti\\
$(x_5 \lor$ ¬$x_6 \lor$ ¬$x_7) \land (x_4 \lor x_5 \lor x_7) \quad \quad \quad ($¬$x_1 \lor$ ¬$x_2 \lor x_3) \land ($¬$x_1, $¬$x_2, $¬$x_3)$\\
\\
La WFF è soddisfacibile se e solo se tutte le sue sottoformule indipendenti sono soddisfacibili.\\
Nel caso in cui siano effettivamente presenti, e detto $m$ il numero di variabili su cui è definita la sottoformula indipendente definita sul maggior numero di variabili, allora la complessità temporale per analizzare l'intera WFF si abbassa a $O(2^m)$.\\
\\
La separazione in sottoformule indipendenti può essere effettuata in tempo polinomiale utilizzando le strutture dati per insiemi disgiunti, in particolare una foresta di insiemi disgiunti \cite{intro_algo}.
L'algoritmo proposto costruisce un nodo per ogni variabile e, analizzando le clausole della WFF, costruisce degli alberi in modo tale che tutte le variabili contenute in un albero siano legate, ma due variabili appartenenti a due alberi diversi non lo siano.\\
Per poter attestare i vantaggi delle strutture dati per insiemi disgiunti, però, bisogna prima comprendere il concetto di \textit{analisi ammortizzata}. Durante l'esecuzione di un algoritmo, un'operazione può essere ripetuta più volte, e le complessità temporali delle differenti chiamate possono variare. La complessità dell'operazione nel caso peggiore sarebbe pari alla peggiore tra tutte le complessità delle diverse esecuzioni, mentre quella dell'intera sequenza sarebbe pari al prodotto tra la complessità peggiore appena considerata e il numero di chiamate effettuate.
In alcuni casi, però, un'analisi di questo tipo può risultare inaccurata: un'operazione potrebbe infatti, considerando l'intera esecuzione dell'algoritmo, mantenere una complessità media più contenuta rispetto a quella calcolata con il classico studio del caso peggiore.
L'analisi ammortizzata prevede quindi di effettuare un'analisi di costo sull'intera sequenza di esecuzioni dell'operazione.\\
\\
Le strutture dati per insiemi disgiunti sono basate su tre operazioni:
\begin{itemize}
    \item \textsc{Make-set(x)}: crea l'insieme disgiunto in cui l'unico elemento contenuto, che è anche rappresentante, è \textsc{x}. Ha complessità $O(1)$.
    \item \textsc{Find-set(x)}: trova il rappresentante dell'insieme disgiunto in cui è  contenuto \textsc{x}. Ha costo ammortizzato $O(1)$.
    \item \textsc{Union(x, y)}: unisce i due insiemi disgiunti i cui due rappresentanti sono rispettivamente \textsc{x} e \textsc{y}. Ha costo ammortizzato $O(1)$.
\end{itemize}
L'utilizzo di strutture dati per insieme disgiunti è quindi molto efficiente.\\
Siano $n$ il numero di operazioni \textsc{Make-set}, $n-1$ il numero di operazioni \textsc{Union} e $f$ il numero di operazioni \textsc{Find-set}, allora la complessità totale per la creazione degli insiemi disgiunti è pari a: 
\begin{center}
    $\Theta(m \; \alpha(n))$
\end{center}
con $m$ numero di operazioni su insiemi disgiunti effettuate, $n$ numero di elementi (nel caso di SAT il numero di variabili) e $\alpha(n)$ una funzione che cresce molto lentamente.
In qualsiasi applicazione delle strutture dati per insiemi disgiunti, $\alpha(n) \le 4$ e quindi si può considerare costante. Di conseguenza, la complessità è lineare in $m$.\\
Il numero di operazioni su insiemi disgiunti eseguite dall'algoritmo di separazione in sotto-WFF indipendenti cresce però in maniera polinomiale rispetto al numero di variabili, e di conseguenza la complessità temporale della creazione degli insiemi disgiunti è pari a:
\begin{center}
    $O(n^k)$
\end{center}
Sia la WFF definita su clausole con esattamente $k$ letterali; supponendo di aver già 
effettuato l'accorciamento della WFF in fase di preprocessing (ovvero di aver rimosso 
tutti i bit rappresentanti clausole con almeno un letterale su una variabile non presente nella WFF), 
allora si ha la garanzia che ogni variabile sia vincolata ad almeno $k-1$ variabili, e che 
quindi ogni sottoformula indipendente sia definita su un numero di variabili maggiore di 
$k-1$.\\
Inoltre, il numero di sottoformule indipendenti è sicuramente minore o uguale al numero di clausole presenti nella WFF.
\begin{algorithm}
    \caption{Algoritmo per dividere la WFF in sottoformule indipendenti.}
    \label{alg:split_indipendent_wff}
    \begin{algorithmic}[1]
        \Procedure{Split-indipendent-wff}{wff}
        \State nodes $\gets$ nuovo array di puntatori ai nodi sulle variabili
        \For{i $\gets$ 0 to wff.varNum - 1}
            \State nodes[i] $\gets$ \Call{Make-set}{i}
        \EndFor\\

        \State tmpClause $\gets$ \Call{Construct-clause}{wff.kSat}
        \For{i $\gets$ 0 to wff.length}
            \For{j $\gets$ 0 to clause.length - 2}
                \State firstRep $\gets$ \Call{Find-set}{nodes[clause.vars[j]]}
                \State secondRep $\gets$ \Call{Find-set}{nodes[clause.vars[j + 1]]}
                \If{firstRep $\neq$ secondRep}
                    \State \Call{Union}{nodes[clause.vars[j]], nodes[clause.vars[j + 1]]}
                \EndIf
            \EndFor
            \State \Call{Increment-clause}{tmpClause, wff.varNum}
        \EndFor\\

        \State ... construct wffs from disjoint set ...
        \EndProcedure
    \end{algorithmic}
\end{algorithm}

\begin{algorithm}[H]
    \caption{Operazioni su insiemi disgiunti \cite{intro_algo}}
    \begin{algorithmic}[1]
        \Procedure{Make-set}{x}
            \State node $\gets$ nuovo nodo
            \State node.p $\gets$ x
            \State node.rank $\gets$ 0
            \State \Return node
        \EndProcedure\\

        \Procedure{Union}{x, y}
            \State firstRep $\gets$ \Call{Find-set}{x}
            \State secondRep $\gets$ \Call{Find-set}{y}
            \State \Call{Link}{firstRep, secondRep}
        \EndProcedure\\

        \Procedure{Link}{x, y}
            \If{x.rank $>$ y.rank}
                \State y.p $\gets$ x
            \Else
                \State x.p $\gets$ y
                \If{x.rank = y.rank}
                    \State y.rank $\gets$ y.rank + 1
                \EndIf
            \EndIf
        \EndProcedure\\

        \Procedure{Find-set}{x}
            \If{x $\neq$ x.p}
                \State x.p $\gets$ \Call{Find-set}{x.p}
            \EndIf
            \State \Return x.p
        \EndProcedure
    \end{algorithmic}
\end{algorithm}